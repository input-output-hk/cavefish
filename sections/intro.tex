% !TEX root = ../lc_main.tex

\maketitle

\begin{abstract}
Blockchain light clients (LCs) are users with limited resources who cannot maintain a fully validated local copy of the ledger state. Consequently, they rely on service providers (SPs), typically full nodes, to access the data needed for tasks such as transaction construction or interacting with off-chain applications.
In this work, we propose a protocol and model for UTxO-based platforms that enables LCs to submit transactions with minimal local storage and computation. Our solution is a two-party computation (2PC) protocol between an LC and an SP. The LC instructs the SP to create a transaction according to its specifications, but only a blinded version of the result is shared. The blinded form prevents the LC from altering the transaction or constructing a new (valid) transaction, while still allowing the LC to verify that the transaction satisfies their original specification.
To achieve this, we introduce a secure predicate mechanism and a weakly blind signature scheme. This enables the SP to obtain a valid signature on the original (unblinded) transaction from the LC, which can then be submitted to the network. The result is a trustless interaction where the LC achieves their transaction goal, and the SP receives compensation for their work.
To optimize communication and computational overhead, we design an extension of Secure Blind Schnorr Signatures called Weakly Blind Predicate Signatures, which relaxes the unlinkability requirement in standard blind signatures.
Lastly, we implement and benchmark the Non-interactive Argument of Knowledge (NArg) component of our protocol on two major UTxO-based blockchain platforms. Despite this component being the most computationally intensive part, our evaluation demonstrates that proving and verification times, as well as associated circuit sizes, are well within practical bounds for real-world deployment.
\textcolor{orange}{play up communication optimality: LC does not read blockchain. rationality }


%Blockchain light clients (LCs) are platform users that do not have the capacity to locally maintain current,
%fully-validated ledger state. For this reason, light clients rely on service providers (SPs, which are full nodes) for obtaining the blockchain data they require, e.g. for transaction construction or running off-chain apps.
%In this paper, we propose protocol and model in which a light client of a UTxO-style platform has the goal of submitting a transaction with minimal local state and minimal local processing. We propose a 2-party computation protocol between the LC and the SP. First, the light client creates a specification for the transaction they would like the SP to construct, then the SP builds such a transaction, while including a small payment to cover the cost of computation.
%The SP sends the LC a blinded version of the constructed transaction for signing and checking. It is modified such that it is impossible for the LC to use the provided data to construct a distinct new (valid) transaction without guessing pre-images of transaction hashes, but has enough information to allow the LC to
%check that it meets their specification. Using a secure predicate and partially blind signature scheme, the SP is able to obtain a signature on the original (unmodified) transaction,
%which is submitted to the network for inclusion in a block. This 2-PC protocol constitutes a trustless
%interaction between the LC and the SP resulting in the LC's desired transaction being applied
%to the ledger state, and the SP receiving compensation for their work.

%In order to realize our light client protocol in a communication optimal way, we develop an adaptation of Secure Blind Schnorr Signatures that we call Weakly Blind Predicate Signatures, alleviating the unlinkability requirement of standard Blind Schnorr Signatures.

%Finally, we implement and benchmark the Non-interactive Argument of Knowledge as part of our protocol and constitutes the most time and resource intensive component of our construction.
\end{abstract}



\section{Introduction}
Blockchain technologies have emerged as a foundational component of decentralized systems, offering strong guarantees of data integrity, censorship resistance, and fault tolerance through cryptographic protocols and distributed consensus. Within this domain, the Unspent Transaction Output (UTxO) model represents a distinctive paradigm for managing asset ownership and validating transactions. 
The UTxO model was initially introduced by Bitcoin and subsequently adopted by other platforms such as Cardano.
In contrast to account-based models, UTxO-based blockchains employ a stateless transactions that facilitate parallelism and improve auditability, but also introduce challenges for client verification.

Full nodes in a UTxO-based blockchain are required to download and validate the entire chain history to ensure correctness and security. This requirement presents a significant barrier to participation for resource-constrained devices, such as smartphones and embedded systems. Light client (LC) protocols aim to mitigate this issue by enabling nodes to interact with the blockchain in a secure and efficient manner without maintaining full historical data. These protocols must strike a careful balance between minimizing resource consumption and preserving critical security properties, such as transaction inclusion, double-spending resistance, and above all, chain validity.

Blockchains are append-only data structures that grow continuously over time. As the chain length increases, it becomes prohibitively expensive for a light client (LC) to scan the entire history to verify past transactions or to locate a specific UTxO. In UTxO-based systems, where each transaction consumes and produces discrete outputs without a centralized account state, the ability to efficiently access historical transaction data becomes essential. Without mechanisms to support succinct historical queries or proofs of inclusion, LCs may be forced to rely on third-party services or sacrifice security assumptions.

At the core of the protocol presented in this paper is the question of ``how can a user of a light client request and approve payments (e.g. from their wallet) in a secure and decentralized way without knowing anything about the current chain and ledger state and minimal communication effort?'' 
In this paper, we present solution to this question in the form of a novel intent-based light client protocol designed for UTxO-based blockchains.
Our solution enables LCs to query the current ledger state and submit transactions requiring only minimal local storage and computation. In order to submit a transaction on chain, the LC engages with a service provider (SP) in a two-party computation protocol that yields a signed transaction indistiguishable from one created by a full node. In addition to the low storage and computational requirements, our LC protocol is communication-optimal. After the LC has instructed the SP about the type of transaction it wishes to create, the protocol can be completed in as few as two rounds. The LC does not need to download or let alone parse the blockchain. The only information the SP must obtain from the LC are the addresses where the funds are located.
If the LC uses a hierarchical wallet that allows straightforward address discovery, only the LC's (master) public key together with the chain code are required.
These are sufficient for the SP to scan the UTxO ledger and construct a transaction according to the specifications requested by the LC.
The only missing part are the signatures by the LC before the transaction can be posted on chain. But instead of sharing the created transaction with the LC verbatim and obtaining the missing signatures, the SP transmits the result in blinded form, which is what we call an abstract transaction. The blinded form of the transaction prevents the LC from modifying the transaction and/or posting the transaction to the chain on its own.

The LC and SP then complete a blind signature protocol where the LC verifies if the transaction is according to its defined specifications and only then creates valid signature(s) which are sent back to the SP.
As a last step, the SP attaches the signatures to the transaction and posts them to the blockchain on behalf of the LC.

To make our scheme viable in the real-world, we adapt the blind Schnorr signature scheme in~\cite{blindsigs}.
More precisely, we omit the unlinkability requirement once the transaction has been published on the blockchain.
In a permission-less blockchain system, the ledger is publicly accessable, and therefore, the transaction signed by the LC has to stay \emph{private until posted} only.
This insight allows us to turn the construction into a weakly blind signature scheme, a simplification that reduces the space and time complexity of the zero-knowledge component which asserts that the abstract transaction meets the LC's specifications.

In addition to the low communication overhead, our light client protocol gives the SP the ability to be reimbursed for its computation time required to construct the transaction without the need of an additional protocol or exchange. The SP embeds a small fee into the transaction, i.e., an additional UTxO output, which makes up for the SP's costs and small reward. If the LC does not agree with the (amount of the) fee, it simply does not provide its signatures at the last step of the blind signature protocol.
This forces the SP to charge a realistic fee for its services as otherwise the LC approaches a different SP.

To summarize, our light client protocol constitutes a completely trustless interaction between a light client 
and service provider that differs from existing approaches as it does not aim to synchornize the LC to the current ledger state despite its resource contraints. 
Since our protocol features atomicity of service and payment, no enrolment or other set up phase is required,  letting the light client freely choose any suitable service provider.

Our work can be implemented for any UTxO or EUTxO blockchain provided the intent specification is adapted to capture the ledger model and the blockchain supports standard Schnorr signatures. We implement and benchmark the zero-knowledge component of our scheme for two major blockchain platforms, Bitcoin and Cardano, and show that even an ``unoptimized'' implementation achieves proving and verification times that are viable in a real-world deployment.
\textcolor{orange}{mr: needs 1-2 passes and some references.}

%\todobox{
%What sets us apart?
%
%\begin{itemize}
%    \item Atomicity of payment+service
%
%    \item permission-less, decentralized
%
%    \item Model for (trustless) 2-party transaction construction rather than proving things about chain/ledger state
%
%    \item Do not require establishing a relationship with SP or any other set-up
%
%    \item inherent timeliness of transaction construction incentivized by SPs desired to earn
%    their tip. This is in contrast with the possibility of stale info provided from old Mithril snapshots in other LC models
%\end{itemize}}


